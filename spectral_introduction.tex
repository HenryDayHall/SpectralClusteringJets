\section{Introduction}\label{sec:JetClustering}

The preferred choice for jet clustering tends to be one of three algorithms;
the \antikt{} algorithm~\cite{Cacciari2008akt}, the Cambridge-Aachen algorithm~\cite{Wobisch1998caJet} and the kt algorithm~\cite{Ellis1993ktJet}.
They have been the default choice for some time because they have a number of desirable properties.
They are infrared safe, excellent implementations of them are available (see \fastjet{}~\cite{Cacciari2011FastJet})
and it is flexible enough to capture many signals with minimal parameter changes.
These algorithms are recursive and agglomerative.
A recursive algorithm is well suited to clustering objects when the number of groups is not known at outset.
Agglomerative algorithms are easier to design in a manner that is infrared safe,
as collinear particles tend to combine early due to their small angle,
and soft particles tend to combine with hard particles rather than clustering among themselves.

Finding a clustering method that compares favourably to these algorithms is challenging.
Spectral clustering is a candidate that has had considerable success in other studies.
In fluid dynamics spectral clustering has been used to identify the motion
of vortices~\cite{hadjighasem2016votex}, finding that it is possible
to successfully identify the vortex structures in cases with less data available.
It was also seen that spectral clustering was proficient at determining the correct number
of clusters to be found in the fluid.
To reduce the risk of blackouts, power grids may be subdivided into `islands'.
The ideal allocation is found by minimising power flow between islands,
and it was shown in~\cite{fennelly2014power} that spectral clustering
can produce a good solution in less time than other algorithms commonly used for this problem.

%Adaptive spectral clustering with application to tripeptide conformation analysis~\cite{haack2013AdaptiveSC}.  %TODO

To the author's knowledge this clustering algorithm has not yet been applied to jet physics, %TODO - double check.
however, given it's recursive, agglomerative form, it could be a good fit.
This is a non standard approach to spectral clustering; while the embedding step relaxes an optimisation objective,
the agglomerative step does not, this is described in more depth in the next section.
