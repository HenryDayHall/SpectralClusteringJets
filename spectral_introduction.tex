\section{Introduction}\label{sec:JetClustering}

The preferred choice for jet clustering in the context of hadron collider physics tends to be one of three algorithms:  
the \antikt{}~\cite{Cacciari:2008gp, Catani:1993hr, Moretti:1998qx},
the Cambridge-Aachen~\cite{Dokshitzer:1997in,Wobisch:1998wt} or the $k_T$  one~\cite{Ellis:1993tq},
all of these having seen  their origin in $e^+e^-$ physics, see  Refs.~\cite{Sterman:1977wj,Bethke:1991wk,Catani:1991hj,Moretti:1998qx}.
They have been the default choice for some time because they have a number of desirable properties.
They are infrared safe, excellent implementations of them are publicly available (see \fastjet{}~\cite{Cacciari:2011ma})
and they are flexible enough to capture many different jet signals with minimal parameter changes.
These algorithms are recursive (or iterative) and agglomerative.
A recursive algorithm is well suited to clustering objects when the number of groups is not known from the outset.
Agglomerative algorithms create jets by grouping objects,
starting from individual particles,
and continuing to combine the groups of particles into larger groups,
until the desired jet size is reached.
Creating jets that are  IR safe
can be achieved by ensuring that pairs of  particles emerging from soft or collinear emissions, combine early in this process.
Once these IR splittings have been recombined they
cannot influence the rest of the clustering process.

Jet definition precedes further algorithmic methods to extract useful
physical quantities. Finding an alternative clustering method that compares favourably to
these popular jet algorithms, and which offers additional features for further analysis, is our goal.
Success in obtaining clusters based on informative transformations of the data
offers the possibility of exploiting such representations.
In this paper, we use Laplacian eigenmaps \cite{Belkin:2003_unfound4} to represent the particles
in an event, a procedure employed in applications such as image segmentation \cite{Shi:1997_unfound595}
and called spectral clustering \cite{Ng:2001_unfound543}.
Spectral clustering has also had success in other physics contexts, such as to identify the motion
of vortices~\cite{Hadjighasem:2016_unfound447} in fluid dynamics,
to determine the correct number of clusters to contain the vortices.
Furthermore, to reduce the risk of blackouts, power grids may be subdivided into `islands',
which are
electromechanically stable regions with minimum load shedding.
The ideal location of such islands is found by minimising the power flow between them using spectral clustering as shown in~\cite{HaoLi:2005_unfound114}.  A hierarchical, agglomerative algorithm for the same was introduced in~\cite{RJSanchezGarcia:2014_unfound420}.
This agglomerative approach is what we show in this paper to be suitable also in the context of jet physics.

The plan of this paper is as follows. In the next section, we will introduce the fundamentals of the theory of spectral clustering. In the following one, we will describe the details of the specific method that we have applied. The numerical results will then follow. Finally, we will draw our conclusions. 
