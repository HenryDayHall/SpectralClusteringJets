\section{Presentations}\label{sec:presentations}

% GRADnet winterschool
\tocless\subsection{18/02/2018, GRADnet winter school}
Workshop based school on leadership skills and team structuring. 
Had a panel of experts from tech fields.
Covered both theory and practice of good leadership.

% ISSP EMFCSC
\tocless\subsection{14/06/2018 56th EMISSP (ISSP EMFCSC)}
This was a summer school hosted by the EMFCSC (http://www.emfcsc.infn.it/issp) on subnuclear physics.
It covered both theoretical and experimental developments.

In particular there was experimental discussion of IceCube, LIGO, ALICE and LHCb among others.

% MLAI in london
\tocless\subsection{17/07/2018 STFC's Summer School in Artificial Intelligence and Machine Learning}
Summer school hosted by UCL on AI and ML with a focus on physics phenomenology.
A number of industry talks also, the conference being financed by industry.

Poster session was particularly useful, many posters on similar projects and discussion of challenges and drawbacks of ML.
I presented a poster on my attempts to train CSVv2 and DeepCSV replicas.
The questions and advice from other attendees offered valuable new perspectives.

% CMS meeting in bristol
\tocless\subsection{11/07/2018 CMS UK meeting}
Meeting of the UK branch of the CMS collaboration, hosted by the university of Bristol.
Recent developments and analysis are discussed.
Many talks on current ML techniques used at CMS.

I gave a talk on my attempt to replicate CSVv2/DeepCSV.
Again the questions put by the audience offered valuable insight.

% BUSSTEPP
\tocless\subsection{20/08/2018 48th British Universities Summer School in Theoretical Elementary Particle Physics}
Summer school for particle physics hosted by the university of Oxford. 
The school presented basic material on many areas including QFT, lattice, strings and CFT.
Some recent experimental work was also discussed.

I gave a short overview talk here about ML and neural networks.

% SPice of flavour
\tocless\subsection{27/10/2018 UK HEP Forum 2018: The Spice of flavour}
Conference on flavour physics. Predominately experimental discussion with some theory.

%ATM/YTF
\tocless\subsection{19/12/2018, Annual theory meeting and young theorists meeting}
This pair of back to back conferences explores recent theoretical developments in various fields and
Offers young researches a chance to present their work.
It was interesting to learn about similar projects going on across other areas of physics.

% YETI
\tocless\subsection{6/01/2019, YETI}
 Promotes interaction between theorists and experimentalists at the early career stage and to encourage interest in phenomenology.
 This year the topic was proposed collider plans.
 Among the workshops offered there was some practical experience with analysis tools on root files.

%SOTSEF outreach
 \tocless\subsection{16/3/2019, SOTSEF outreach event}
 In collaboration with other students presented an engagement show at the Southampton Science and Engineering Festival (SOTSEF).
 The show was an accessible explanation of the experimental components of a particle collider such as the LHC.
 It described the major components and used bench top interactive experiments to demonstrate the principles behind them.

% Future of particle physics in a post higgs landscape
 \tocless\subsection{4/4/2019, Future of particle physics in a post Higgs landscape}
 Student lead discussion of the field of particle physics in the absence of major discoveries since the Higgs Boson in 2012.
 Lots of work on the extended Higgs sector is presented here.
 I gave a talk on the use of more complicated neural network architectures to classify particle jets.

%Gradnet summer school
 \tocless\subsection{1/7/2019, GRADnet Summer School}
 Summer school on opportunities for physics PhD student in industry.
 It was a useful thing to attend, although most of the options seem rather less exciting than work in academia.

% NExT PHD school
 \tocless\subsection{8/7/2019, Decoding new physics from data: connecting theory and signatures}
 Pedagogical workshop on phenomenology and the impact of new observations on current theories. 
 A very useful overview of many upcoming directions.
 Also a clear an accessible dissection of key statistical concepts and common foibles was presented by Prof. Glen Cowan.

 I presented a poster on potential uses of RNTNs to classify the flavour of jets.

 \tocless\subsection{23/9/2019, 1st Mediterranean Conference in Higgs Physics}
 Conference on experimental and theoretical developments in Higgs Physics.
 I gave a talk on tools for determining parameter spaces.

 \tocless\subsection{29/4/2019, NEXT spring meeting}
 Theoretical and experimental physics talks from members of the NEXT network.
 I gave a talk on Spectral clustering.
