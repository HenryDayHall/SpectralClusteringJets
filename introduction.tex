\section{Introduction}
The investigations presented here straddle a number of topics, machine learning, high energy experimental physics and high energy phenomenology. 
The focus is tools that form and classify jets in challenging topologies.
Such topologies represent hard to access areas of many predictions, such as those of the two Higgs doublet model.
Tools that were able to utilise more of the information held in the data generated by high energy experiments, such as those at the LHC, it might be able to exclude or confirm the present of multiple Higgs like particles.

A particle jet is a particular clustering of observed tracks in a detector.
The clustering is designed to reflect the structure of the event that produced the tracks and it is much needed structure to assist in the processing of the overall event.
The default choice for jet clustering tends to be one of there algorithms;
the anti-kt algorithm~\cite{Cacciari2008akt}, the Cambridge-Aachen algorithm~\cite{Wobisch1998caJet} and the kt algorithm~\cite{Ellis1993ktJet}.
They have been the default choice for some time because they have a number of desirable properties.
They are infrared safe, excellent implementations of them are available (see \fastjet{}~\cite{Cacciari2011FastJet})
and it is flexible enough to capture many signals with minimal parameter change.
These algorithms are recursive and agglomerative.
A recursive algorithm is well suited to clustering objects when the number of groups is not known at outset.
Agglomerative algorithms are easier to design in a manner that is infrared safe,
as they can recombine soft and collinear emissions in early steps.

Once the jets have been formed it then remains to identify the originating particle of the shower.
The originating particle is the particle that leaves the hard interaction, or is immediately descendant of the proton beam,
that decays to form the shower.
There are a number of factors effecting the difficulties of identifying this particle.
One of these would be how well the shower has been isolated by the jet.
Sometimes two showers overlap strongly, so one jet in fact contains the combination of two showers,
identifying both originating particles together is especially challenging.
This is often the case when a very high energy particle decays into lighter particles,
and the descendants have high kinetic energy, which sends them in a boosted configuration as a shallow angle to the beam line.

A signal of particular interest is found in the extended Higgs sector.
Since the discovery of the Higgs Boson in 2012, it's couplings
have been seen to be in agreement with the Standard Model (SM),
however additional Higgs particles remain possible.
One of the simplest extensions to the Higgs sector is the two Higgs doublet model (2HDM)~\cite{Branco2012THDM}.
The second doublet of the 2HDM allows a further 5 particles;
two CP even (\(h\) and \(H\), with, conventionally, \(m_h < m_H\)),
one CP odd (\(A\))
and a pair of charged (\(H^\pm\)) Higgs bosons.

This is relevant to jet physics because the additional Higgs particles tend to decay to \bthing{quark} which hadronize to jets.
These jets may have a highly boosted configuration due to the mass difference between the \bthing{quark} and the heavy Higgs.
As such the 2HDM is a prime example of the types of signals that might benefit from advances in jet formation and classification.
