\section{Conclusions}
Spectral clustering offers a promising new jet formation method.
It is a transparent mechanism with no black box element.
All intermediate steps are interpretable.
It satisfies the need for IRC safety and creates jets with the expected kinematics.

While it has many hyperparameters, then do not appear to be as finely tuned as those of \genkt{}.
This can be seen both in parameter scans, and its adaptability to various datasets.

The adaptability between datasets is remarkable,
a \spectral{} parameter choice tuned on a light higgs cascade
gave excellent performance on both a heavy higgs cascade and a top decay.
On the light higgs cascade \spectral{} gives the correct mass peak,
a narrow mass peak and the highest multiplicity.
This would not be surprising as it was tuned for that dataset.
On the heavy higgs dataset only \antikt{} \(\ktstoppingdeltar{} = 0.8\) 
and \spectral{} give correct mass peaks, and \spectral{} offers considerably higher multiplicities.
This demonstrates that the performance is not dependent on fine tuning and the algorithm is adaptable.

Finally, \spectral{} was applied to a dataset for which the ideal jet radius differed.
Its equivalent parameter \(\sigma_v\) was not allowed to vary to account for this,
instead it was applied again with no parameter changes.
The algorithm again proved to be adaptable and modified its behaviour to follow that of \(\ktstoppingdeltar{} = 0.4\)
without interference.

This is a novel and promising approach to jet formation. 
Initial development already demonstrates flexibility and excellent performance.

