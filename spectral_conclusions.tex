\subsection{Conclusions}
Spectral clustering has the potential to perform at least as well as 
traditional clustering methods such as anti-kt.
Fully re-clustering at each step, as in \spectralfulljet{} is important for good performance,
and thus spectral clustering may prove expensive.

Spectral clustering is certainly capable of producing sharp mass peaks,
though it tends to include more background than is ideal.

\subsubsection{Going forward}
From here, there are three further questions that seem pressing;
\begin{enumerate}
    \item Is spectral clustering IRC safe? Ideally this would be tested 
        by generating samples with soft radiation and collinear splitting
        and comparing the clusters formed with and without these changes.
        Shape variables might also be of assistance here, as they have
        infra-red and collinear limits.
    \item Does spectral clustering perform well on other data samples?
        Comparing the performance of methods with the same hyperparameters on
        different data samples would explore how well the clustering generalises.
        It is important to determine if the method can reliably gather
        any signal, or if it has been tuned to gather the signal we expect to find.
    \item Does the additional information produced by spectral clustering 
        provide any guidance in jet classification?
        Jet classification will be discussed more in section~\ref{sec:jetclasification},
        however, it is worth noting at this point that spectral clustering
        effectively generates additional values associated with each jet,
        and perhaps these values would be of use in classifying the jet.
\end{enumerate}

