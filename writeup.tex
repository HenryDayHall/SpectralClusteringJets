\documentclass{article}
\usepackage{geometry}
\newgeometry{vmargin={18mm}, hmargin={20mm,20mm}}
\usepackage[export]{adjustbox}[2011/08/13]
\usepackage{subcaption}
\usepackage{placeins}
\usepackage[backend=biber, giveninits=true]{biblatex}
\usepackage{graphicx}
\usepackage{color}
\usepackage{listings}
\lstset{basicstyle=\ttfamily}
\usepackage{caption}
\usepackage{amsmath}
\usepackage{amssymb}
\usepackage{amsthm}
\usepackage{siunitx} %This is currently a prealpha version that I got by accident
\usepackage{dcolumn}% Align table columns on decimal point
\usepackage{latexsym}
\usepackage{bm}
\usepackage{multirow}
\usepackage{cancel}
\usepackage{float}
\usepackage{booktabs}
\usepackage{rotating}
\usepackage[toc,page]{appendix}


\let\vec\mathbf
% if we created names here we can quickly change capitalisation etc.
% varient of delta R
\newcommand{\stoppingdeltar}{\ensuremath{R}}
\newcommand{\ktstoppingdeltar}{\ensuremath{R_\text{kt}}}
\newcommand{\distancedeltar}{\ensuremath{\Delta R}}
%b related things, so taht we always get the math-mode right
\newcommand{\beau}{\ensuremath{b}}
\newcommand{\bbar}{\ensuremath{\bar{b}}}
\newcommand{\bthing}[1]{\ensuremath{b\text{-#1}}}
% some names for consistent capitalisation
\newcommand{\genkt}{generalised KT}
\newcommand{\antikt}{anti-KT}
\newcommand{\itercone}{iterative cone}
\newcommand{\spectral}{spectral}
\newcommand{\fastjet}{\lstinline{FastJet}}

\bibliography{writeup}
\graphicspath{{./graphics/}}
\begin{document}
\title{Spectral clustering for Jet physics}
% author list???
%\author{S. Dasmahapatra, H.A. Day-Hall, B. Ford, S. Moretti, C.H. Shepherd-Themistocleous}
	
	\maketitle
	
%	\tableofcontents
    \FloatBarrier
    \begin{abstract}
        An alternative method of jet clustering, utilising Spectral Clustering, is considered.
        The performance of this algorithm on a \(H_{125\text{GeV}} \rightarrow H_{40\text{GeV}} H_{40\text{GeV}} \rightarrow b \bar{b} b \bar{b}\)
        monte carlo sample is compared to that of the Anti-KT algorithm.
        Measures of infra-red collinear safety are also studied.
	\end{abstract}
     
    \FloatBarrier
    \FloatBarrier
    \section{Jet clustering}\label{sec:JetClustering}
This next section considers a completely novel approach to jet clustering, spectral clustering.

As mentioned previously, the default choice for jet clustering tends to be one of there algorithms;
the anti-kt algorithm~\cite{Cacciari2008akt}, the Cambridge-Aachen algorithm~\cite{Wobisch1998caJet} and the kt algorithm~\cite{Ellis1993ktJet}.
They have been the default choice for some time because they have a number of desirable properties.
They are infrared safe, excellent implementations of them are available (see \fastjet{}~\cite{Cacciari2011FastJet})
and it is flexible enough to capture many signals with minimal parameter change.
These algorithms are recursive and agglomerative.
A recursive algorithm is well suited to clustering objects when the number of groups is not known at outset.
Agglomerative algorithms are easier to design in a manner that is infrared safe,
as they can recombine soft and collinear emissions in early steps.

Finding a clustering method that compares favourably to these algorithms is challenging.
Spectral clustering is a candidate that has had considerable success in other studies.
In fluid dynamics spectral clustering has been used to identify the motion
of vortices~\cite{hadjighasem2016votex}, finding that it is possible
to successfully identify the vortex structures in cases with less data available.
It was also seen that spectral clustering was proficient at determining the correct number
of clusters to be found in the fluid.
To reduce the risk of blackouts, power grids may be subdivided into `islands'.
The ideal allocation is found by minimising power flow between islands,
and it was shown in~\cite{fennelly2014power} that spectral clustering
can produce a good solution in less time than other algorithms commonly used for this problem.

%Adaptive spectral clustering with application to tripeptide conformation analysis~\cite{haack2013AdaptiveSC}.  %TODO

To the authors knowledge this clustering algorithm has not yet been applied to jet physics, %TODO - double check.
however, given its recursive, agglomerative form it could be a good fit.

    \FloatBarrier
    \section{Theory of spectral clustering}\label{sec:spectral_theory}
Gathering collimated emissions of particles is the target of jet formation, so this must be decided by localised information.
A representation of observable particles that preserves and accentuates local information
motivates the Laplacian eigenmap~\cite{Belkin:2003_unfound4} and spectral
clustering~\cite{Ng:2001_unfound543}, so as to lead us to believe that these are suitable tools for jet formation.
An excellent description of the theory behind spectral clustering
can be found in~\cite{UlrikevonLuxburg:2007_unfound52} while a short
summary is given in this section.

Spectral clustering is a method by which a set of points are represented in a new space,
called the embedding space, in which they can be easily clustered.
Coordinates of the
points in the embedding space are expressed in terms of the eigenvectors and eigenvalues
of an associated Laplacian matrix, hence the name.

Input data for spectral clustering must be given as a graph,
which is a set of nodes, in this case representing the particles,
and edges which join nodes together, representing relationship between particles.
%Not all nodes must be connected by edges
%but, for simplicity we begin with a fully connected graph.
The edges may be weighted, that is, a positive number is associated with the edge,
called an affinity.
Affinity represents degree of belief that the nodes connected by the edge should go in the same group:
for jet clustering this will be a degree of belief that the particles came from the same shower.

The theory behind the
construction of the embedding space is a relaxation
of optimising criteria that would best 
partition nodes into separate disconnected subgraphs, by
splitting nodes into groups.
In a standard (non-physics) procedure we would start from points with coordinates,
which should be split into a predetermined number \(s\) of clusters.
The points are represented by nodes of a graph.
The edge of the graph joining node (or point) \(i\) and \(j\) has weight \(a_{i, j}\),
which should grow with the probability of \(i\) and \(j\) being in the same group.

To identify groups for the points the graph is split into subgraphs,
\(G_\indicatoridx{k}\), where \(\indicatoridx{k}=1 \dots s\).
These groups should not split up points which are connected by edges with high affinity,
but it should also avoid groups of very uneven size.
Minimising the NCut objective captures this aim, where NCut is defined as
\begin{equation}
    \text{NCut} = \frac{1}{2}\sum_\indicatoridx{k}\frac{W(G_\indicatoridx{k}, \bar{G_\indicatoridx{k}})}{\text{vol}(G_\indicatoridx{k})},
\end{equation}\label{eqn:cost_function}
where \(W(G_\indicatoridx{k}, \bar{G_\indicatoridx{k}})\) is the sum of all the edge weights that must be dropped
to separate the cluster \(G_\indicatoridx{k}\) from the rest of the graph, \(\bar{G_\indicatoridx{k}}\),
so that \( W(G_\indicatoridx{k}, \bar{G_\indicatoridx{k}}) = \sum_{i \in G_\indicatoridx{k}, j \in \bar{G_\indicatoridx{k}}} a_{i, j} \).
In the denominator, \(\text{vol}(G_\indicatoridx{k}) = \sum_{i \in G_\indicatoridx{k}} \sum_{j} a_{i, j}\) is 
the sum of all affinities connecting to a point in \(G_\indicatoridx{k}\).
This denominator is used to penalise forming small clusters.

In order to determine which point will go in which \(G_\indicatoridx{k}\), a set of indicator vectors must be found.
Membership of cluster \(G_\indicatoridx{k}\) will be recorded in the indicator vector \(h_\indicatoridx{k}\):
\begin{equation}\label{eqn:indicator}
    h_{\indicatoridx{k}\,i}= 
    \begin{cases}
        1/\sqrt{\text{vol}(G_\indicatoridx{k})}& \text{if point } i \in G_\indicatoridx{k} ,\\
        0             & \text{otherwise}.
    \end{cases}
\end{equation}

To find these indicator vectors the graph is represented by the graph Laplacian, \(L\), a square
matrix with as many rows and columns as there are points.
To construct this Laplacian we define two other matrices:
an off diagonal matrix 
\(A_{i, j} = (1 - \delta_{i, j})a_{i, j}\)
and a diagonal matrix
\(D_{i, j} = \delta_{i, j}\sum_q a_{i, q}\).
Then the symmetric Laplacian can be simply written as
\begin{equation}\label{eqn:symmetric_laplacian}
    L = D^{-\frac{1}{2}} (D - A) D^{-\frac{1}{2}}.
\end{equation}

Considering just one cluster, \(G_\indicatoridx{k}\), when the Laplacian is multiplied by its indicator vector,
the result is the term that NCut seeks to minimise for that cluster,
\begin{equation}
    h_\indicatoridx{k}'Lh_\indicatoridx{k} = \frac{1}{\text{vol}(G_\indicatoridx{k})}\sum_{i \in G_\indicatoridx{k}, j \in G_\indicatoridx{k}} \left(\delta_{i, j}\sum_{l} a_{l, i} - a_{i, j} \right) = \frac{W(G_\indicatoridx{k}, \bar{G_\indicatoridx{k}})}{\text{vol}(G_\indicatoridx{k})}.
\end{equation}
To obtain the sum of all the terms, stack the indicator vectors into a matrix,
\( h'_\indicatoridx{k} L h_\indicatoridx{k} = (H'L H)_{\indicatoridx{k}\indicatoridx{k}}\),
and the NCut aim described earlier becomes the trace
\begin{equation} \text{NCut}(G_\indicatoridx{1},G_\indicatoridx{2}, \dots G_\indicatoridx{n}) \equiv \frac{1}{2} \sum_{k=1}^n \frac{W(G_\indicatoridx{k}, \bar{G_\indicatoridx{k}})}{\text{vol}(G_\indicatoridx{k})} = \text{Tr}(H'LH),\end{equation}
where \(H'H = I\).
This is still a Non-deterministic Polynomial (NP-hard) problem~\cite{Leeuwen:1990_unfound0}.
However if we relax the requirements made on \(h\) in Eqn.~(\ref{eqn:indicator}),
allowing the elements of \(h\) to take arbitrary values, then the Rayleigh-Ritz theorem provides a solution.
Trace minimisation in this form is done by finding the eigenvectors of \(L\) with smallest eigenvalues,
\begin{equation}
    \lambda_\text{min} = \min_{\|x\|\ne 0 } \frac{x^H L x}{x^H x},
\end{equation}
where \(x\) is the relaxed indicator vector and an eigenvector of \(L\).
Notice that \(L\) is a real symmetric matrix
and, therefore, all its eigenvalues are real.
Due to the form of the Laplacian, there will be an eigenvector with components all of the same value and its eigenvalue will be \(0\).
This corresponds to the trivial solution of considering all points to be in one group.
The next \(c=s\) eigenvectors of \(L\), sorted by smallest eigenvalue, can be used to allocate points to \(s\) clusters.

These eigenvectors are then used to determine the position of the points in the embedding space.
Each eigenvector has as many elements as there are points to be clustered,
so the coordinates of a point are the corresponding elements of the eigenvectors.
%This is all the information the theory of spectral clustering provides.
%The steps required to make use of this information are not dictated by the theory
%and they must be carefully selected to respect the physics involved.

The standard method above is designed to form a fixed number of clusters,
but typically we do not know how many jets should be created in an event.
We will create an alternative algorithm, beginning with the principles of
spectral clustering and adjusting to the needs of the physics being studied.
Using the positions in embedding space, the points can be gathered agglomeratively,
so that we do not need to choose a predetermined number of clusters.

\subsection{Distance in the embedding space}\label{sec:embedding_distance}
When the relaxed spectral clustering algorithm is used to create an embedding space,
points in each group will be distributed in this embedding space.
Each point can be seen as a vector, its direction  indicating the group to which this point should be assigned.
Changes in magnitude of the vectors cause the Euclidean distance between the corresponding points to grow,
however, an angular distance is invariant to changes in magnitude,
 therefore it  is a suitable measure to use.

\subsection{Information in the eigenvalues}\label{sec:eig_norm}
When the clusters in the data are well separated,
the affinities between groups are close to \(0\)
and the eigenvalues will also be closer to \(0\).
So a small eigenvalue means that the corresponding eigenvector
is separating the particles cleanly according to the affinities.
It is possible to make use of this information.

In a traditional application of spectral clustering, the number of clusters desired, \(s\), is predetermined.
The embedding space is created by taking \(c=s\) eigenvectors with smallest eigenvalues, excluding the trivial eigenvector.
The embedding space then has \(c\) dimensions.

When forming jets we do not know from the outset how many clusters to expect in the dataset,
so the number of eigenvectors to keep is not clear.
We cannot set \(c=s\).
While we could choose a fixed, arbitrary number of eigenvectors, this is suboptimal.
A better approach is to take all non-trivial eigenvectors corresponding to eigenvalues
smaller than some limiting number, \(\lambda_\text{limit}\).
For a symmetric Laplacian the eigenvalues are \(0 \leq \lambda_\eigenidx{1} \leq \lambda_\eigenidx{2} \leq \cdots \lambda_\eigenidx{n} \leq 2\),
and \(\lambda_\eigenidx{k}\) is related to the quality of forming \(\eigenidx{k}\)
clusters~\cite{JamesRLee:2014_unfound736}.
Removing eigenvectors with eigenvalues close to \(0\) would result
in discarding useful information, while retaining eigenvectors 
whose eigenvalues are close to \(2\) would increase the noise.
Values of \(0 < \lambda_{\mathrm{limit}} < 1\) are sensible choices,
and within this range the choice is not critical.
Then, the number of dimensions in the embedding space will vary,
according to the number of non-trivial eigenvectors with corresponding \(\lambda < \lambda_\text{limit}\).

There is one more manipulation from the information in the eigenvalues.
The dimensions of this embedding space are not of equal importance.
This can be accounted for by dividing the eigenvector by some power, \(\beta\), of the eigenvalue.

Let the eigenvectors for which \(\lambda < \lambda_\text{limit}\) be
\begin{equation}
    \sum_j L_{i, j} x_{\eigenidx{n}\,j} = \lambda_\eigenidx{n} x_{\eigenidx{n}\,i}.
\end{equation}
Then, the coordinates of the \(j^\text{th}\) point in the \(c\) dimensional embedding space
become \(m_j = \left(\lambda_\eigenidx{1}^{-\beta} h_{\eigenidx{1}\,j}, \dots \lambda_\eigenidx{c}^{-\beta} h_{\eigenidx{c}\,j},\right)\).
In effect, the magnitude of the vectors, \(m_j\), in the \(n^\text{th}\) dimension are compressed by a factor \(\lambda_\eigenidx{n}^\beta\),
so the larger \(\lambda_\eigenidx{n}\) the greater the compression.

\subsection{Stopping conditions}\label{sec:stopping_condintion}

If a recursive algorithm is to be chosen, like in the \genkt{} algorithm, a stopping condition is needed.
A stopping condition based on smallest distance between points in the embedding space was attempted 
but this was not found to be stable.
Choosing an acceptable value for all events was not possible.

Distance between the last two points to be joined before the desired jets have been formed
 varies significantly between events, so minimum separation is not a good stopping condition.
The average distance between points before this last joining is more stable because it 
is balanced by two opposing influences.
When points are joined together in a fix number of dimensions the average
distance between points rises.
If this were used in physical space it would be roughly proportional to the number of points remaining.
So, in physical space, if we stopped clustering when average distance exceeded some cut-off,
we would expect roughly the same number of jets in each event.
However, the embedding space has a variable number of dimensions.
When lots of clustering still remains to be done the lower eigenvalues mean that
the embedding space has more dimensions,
as described in section~\ref{sec:eig_norm}.
When the number of dimensions in the embedding space falls,
the mean distance between points will also fall.

As points combine the mean distance will rise,
but when fewer combinations with higher affinity remain the number of
dimensions in the embedding space falls, counteracting the rise in mean distance.
In short, the mean distance in the embedding space makes a natural cut-off.

    \FloatBarrier
    \section{Method}
In this section the methodology is covered in four parts.
Firstly, the practical procedure chosen in this work for applying \spectral{} clustering  is given.
Secondly, choices and interpretations for the variable parameters in this algorithm are given.
Thirdly,  the datasets against which this will be measured are specified.
Fourthly, the procedure for checking IR safety is described.

    \begin{figure}[!t]
        \center
        \includegraphics[width=0.7\textwidth]{graphics/embedding_space_simple2.pdf}
        \caption{A single event and its embedding space, as created by \spectral{} clustering.
            At the top the grey plot shows the particles in the event as points on the unrolled detector barrel.
            The colour of each point indicates the shower it came from.
            The lower two plots show the first 4 dimensions of the embedding space,
            and the location of the points within the embedding space.}\label{fig:embedding_space_simple}
    \end{figure}    


For clarity, note that the variable pseudorapidity is never used
in the algorithms proposed; all references to rapidity $y$ correspond to
\begin{equation}\label{eqn:rapidity}
    y = \frac{1}{2} \ln\frac{E + p_z}{E - p_z}.
\end{equation}

Besides rapidity, barrel angle, \(\phi\), is also used as a coordinate.
The barrel angle is the angle of the particle in the plane perpendicular to the beam.
Barrel angle and rapidity form an orthogonal coordinate system.

\subsection{Spectral clustering algorithm}\label{sec:spectralmethodalgo}
    For every simulated event the following process is used to identify the jets
   % should we point to the repo?
   % \footnote{Code available at \url{https://github.com/HenryDayHall/jetTools}}.
    To begin with, relevant cuts are applied to the particles to simulate the detector's
    reconstruction capability.
    (These are described in detail in section~\ref{sec:particle_data}.)
    Then all particles are declared pseudojets  and given an index, \(j = 1 \dots n\), with no particular order.
    The algorithm is agglomerative, recursively selecting pairs of pseudojets to merge;
    hence, the first iteration step is labelled \(t=1\).

    When the two pseudojets to be merged, \(i\) and \(j\), have been identified they are combined
    using the E-scheme.
    The E-scheme forms a new pseudojet by summing the \(4\)-momentum of the two join pseudojets;
    \(p(t+1)_l = p(t)_i + p(t)_j\).
    The steps used to select two pseudojets to merge proceed as follows;


    \begin{enumerate}
        \item \label{step:start} The pseudojets are used to form the nodes of a graph,
        the edges of which will be weighted by some measure of proximity between the particles called affinity.
        To obtain an affinity, first a distance is obtained.
        Between pseudojets \(i\) and \(j\) this would be \(d(t)_{i,j} = \sqrt{(y(t)_i - y(t)_j)^2 + (\phi(t)_i - \phi(t)_j)^2}\)
        where \(y(t)_j\) is the rapidity of pseudojet \(j\) at time step \(t\) and \(\phi(t)_j\) is the barrel angle, likewise for \(i\).
        No \(p_T\) dependence is used, as customary in many traditional jet clustering methods.

    \item \label{step:affinity} The affinity must increase as pseudojets become more similar,
        whereas the distance will shrink.
        We chose \(a(t)_{i,j} = \text{exp}(-d(t)_{i,j}^\alpha/\sigma_v)\),
        where $\alpha=2$ is the standard Gaussian kernel as
        used in~\cite{Belkin:2003_unfound4}.
            Distances much larger than \(\sigma_v\) are only allowed very small affinities,
            thus less influence over the clustering.

    \item\label{step:KNN} Pseudojets that are far apart have low affinity,
        hence are unlikely to be good candidates for combination.
        Removing these affinities reduces noise.
    A fixed number, \(k_\text{NN}\), of neighbours of each pseudojet is 
    preserved while all other affinities are set to zero.
    Thus, in a group with more than \(k_\text{NN}\) pseudojets,
    each pseudojet has at least \(k_\text{NN}\) non-zero affinities with other pseudojets.

\item\label{step:laplacean} These affinities allow the construction of the symmetric normalised
        Laplacian; which is proportional to \(-a(t)_{i, j}\)
        in the \(i\)$^{\rm th}$ row and \(j\)$^{\rm th}$ column and exactly \(1\) on the diagonal.
        For ease of notation, let \(z(t)_j\) be a measure of the size a pseudojet \(j\) contributes to a cluster,
        with \(z(1)_j = \sum_k a_{i,k}\).
        Then define square matrices \(A(t)_{i, j} = (1 - \delta_{i, j}) a(t)_{i, j}\) and \(Z(t)_{i, j} = \delta_{i, j} z(t)_i\)
        The Laplacian can now be written as
       \begin{equation}\label{eqn:Laplacian}
        L(t) = Z(t)^{-\frac{1}{2}}(Z - A)Z(t)^{-\frac{1}{2}}
       \end{equation}
        After each step this Laplacian shrinks by one row and column.
        When two pseudojets have been combined, instead of calculating \(z_j\) as the sum of the affinities of the combined pseudojet,
        the new \(z_j\) is the sum of the two previous \(z_j\)'s.
        For example, if pseudojets \(1\) and \(2\) from \(t=1\) are to be combined to make pseudojet \(1\) in \(t=2\),
        then \(z(2)_{1} = z(1)_1 + z(1)_2\) rather than the sum of affinities between the new pseudojet \(1\) and other pseudojets in step \(t=2\).
        This condition is seen to be required for IR safety. 

        As such, after the first time step, \(L\) will no longer be a proper graph Laplacian.
        Its rows and columns do not sum to zero.
        However, this new \(L\) appears to maintains similar behaviour
        to a propper Laplacian in the embedding space it creates.

    \item \label{step:eigenvectors} The eigenvectors of $L(t)$, ($q$ the eigenvalue index)
            \begin{equation}
                L(t) h(t)_q = \lambda(t)_q h(t)_q,  \; q=1, \ldots, c
            \end{equation}
            	are used to create the embedding of the pseudojets.
            The eigenvector corrisponding to the smallest eigenvalue represents the trivial solution,
            that places all points in the same cluster (see Sec.~\ref{sec:spectral_theory}).
            All non-trivial eigenvectors, corresponding
            to eigenvalues less that an eigenvalue limit \(\lambda(t)_c < \lambda_\text{limit} < \lambda(t)_{c+1}\)
            are retained. See section~\ref{sec:eig_norm}.
            {NOTE: Eigenvalues $0\leq \lambda(1)\leq 2$.
            However, $\lambda(t)$ for $t>1$ are bounded but in a different range.} 

        \item \label{step:compression} A eigenvector is divided by the corresponding eigenvalue raised to \(\beta\).
            This acts to compress the dimensions that hold less information, again, see section~\ref{sec:eig_norm}.
            The embedding space can now be formed.
            The eigenvectors have as many elements as there are pseudojets and the coordinates of
            the \(j^\text{th}\) pseudojet at time step \(t\)
            are defined to be
            \(m(t)_j = \left(\lambda_1(t)^{-\beta} h_1(t)_j, \dots \lambda_c(t)^{-\beta} h_c(t)_j\right)\).
            %The first embedding space of a clustering is shown in Fig.~\ref{fig:embedding_space_simple}.

        \item  A measure of distance between all pseudojets in the embedding space is calculated.
            In the embedding space angular distances are most appropriate (see section~\ref{sec:embedding_distance}):
            \begin{equation}
                d'(t)_{i, j} = \arccos\left|\frac{m(t)_i\cdot m(t)_j}{\|m(t)_i\| \|m(t)_j\|}\right|.
            \end{equation}
            where \(\|m\|\) is the (Euclidean) length of \(m\).

        \item\label{step:stoppingcondition}

            Provided the mean of this distance is less than \stoppingdeltar{}, that is,
            \begin{equation}
                \frac{2}{c(c-1)}\sum_{i\ne j} d'(t)_{i, j} < \stoppingdeltar{},
            \end{equation}
            then the two pseudojets that have the smallest embedding distance are combined.
            NOTE: there are $\frac{c(c-1)}{2}$ possible pairs, 
            where \(c\) is the number of pseudojets remaining.
            (Reasons for this stopping condition are given in section~\ref{sec:stopping_condintion}.)
        
     \end{enumerate}
    %  
    When the mean of the distances in the embedding space rises above \stoppingdeltar{},
    then all remaining pseudojets are promoted to jets.
    Jets with less than 2 tracks are removed and their contents considered noise.
    Further cuts may then be applied as described in section~\ref{sec:particle_data}.

    These steps will form a variable number of jets from a variable number of particles.
    An example of the constructed first embedding space is shown in Fig.~\ref{fig:embedding_space_simple}.
    This illustrates how the embedding space highlights the clusters.

\subsection{Tunable parameters}\label{sec:spectralmethodparam}
Unlike most Machine Learning (ML) techniques, \spectral{} clustering does not have large arrays of learnt parameters.
The parameters for the clustering are a small, interpretable  set.
Appropriate values were chosen by performing scans and observing the influence of changes to the parameters on jets formed.

In section~\ref{sec:spectralmethodalgo}, 6 parameters are named:
\(\sigma_v\), \(\alpha\), \(k_\text{NN}\), \(\lambda_\text{limit}\), \(\beta\) and \stoppingdeltar{}.
These parameters have a range of values for which sensible results are obtained.
The interpretation of these parameters is as follows.
\begin{itemize}
    \item \(\sigma_v\): introduced in step~\ref{step:affinity}, this is a scale parameter in physical space.
                      The value indicates an approximate average distance for particles in the same shower,
                      or alternatively, the size of the neighbourhood of each particle.
                      It is closely tied to the stopping parameter for the \genkt{} algorithm,
                      \ktstoppingdeltar{},
                      they both relate to the width of the jets formed.
                      It should take values on the same order of magnitude as \ktstoppingdeltar{},
                      \sout{, i.e.,  of \(\mathcal{O} (0.1)\)}.
    \item  \(\alpha\): also introduced in step~\ref{step:affinity},
           this changes the shape of the distribution used to describe the neighbourhood of a particle.
           Higher values reduces the probability of joining particles outside \(\sigma_v\);
           \(\alpha=2\) defines a Gaussian kernel.
       \item \(k_\text{NN}\): introduced in step~\ref{step:KNN}, this dictates the minimum number of non-zero affinities around each point.
           Lower values create a sparser affinity matrix, reducing noise at the potential cost of lost signal.
           Values above \(7\) are seen to have little impact.
       \item  \(\lambda_\text{limit}\): introduced in step~\ref{step:eigenvectors}, is a means of limiting the number of eigenvectors used
           to create dimensions in the embedding space.
           Only eigenvectors corresponding to eigenvalues less than \(\lambda_\text{limit}\) are used.
           Thus, the number of dimensions in the embedding space can be increased with a larger \(\lambda_\text{limit}\). 
           However, as the eigenvalues will be influenced by the number of clear clusters available, 
           there will not be the same number of dimensions in each event.
           (This is discussed in section~\ref{sec:eig_norm}.)
           For a symmetric Laplacian the eigenvalues are \(0 \leq \lambda_1 \leq \lambda_2 \leq \cdots \lambda_n \leq 2\),
           and \(\lambda_k\) is related to the quality of forming \(k\)
           clusters \cite{JamesRLee:2014_unfound736},
           so values of $\lambda_{\mathrm limit} < 1$ are sensible choices.
       \item  \(\beta\): introduced in step~\ref{step:compression}, it 
          accounts for variable quality of information in the eigenvectors, as given by their eigenvalues,
        in such a way that the dimensions of the embedding spaces 
        corresponding to higher eigenvalues are compressed.
        in such a way that the dimensions of the embedding spaces with lower
        quality information are compressed.
        The larger the value of \(\beta\) the more dimensions with
        lower quality information are compressed.
        (This is discussed in section~\ref{sec:eig_norm}.)
%        This is should be \(\mathcal{O}(1)\).
    \item \stoppingdeltar{}: is introduced in step~\ref{step:stoppingcondition}, it
         determines the expected spacing between jets in the embedding space.
         As the number of dimensions in the embedding space grows with increasing 
         number of clear clusters, it will not result in the same or
         similar number of clusters each time.

\end{itemize}


To investigate the behaviour of the clustering when the parameters change, scans where performed.
On a small sample of 2000 events  the clustering is performed with many different parameter choices.

With the aid of MC truth information a metric of success can be created.
For each object we wish to find (e.g., a \bthing{quark}) 
the MC truth can reveal which of the particles that are visible to the detector have
been created by that object.
In many cases, a particle seen in the detector will have been created by two objects,
such as a particle coming from an interaction between a \(b\bar{b}\) pair,
in these cases both objects are considered together.
The complete set of visible particles that came from these objects could be referred to as their descendants.
The aim in jet clustering is to capture only all of the descendants in the same number of jets as there were objects that created them.
So the descendants of a \(b\bar{b}\) pair should be captured in exactly 2 jets.

    \begin{figure}[!t]
        \begin{minipage}[c]{0.6\textwidth}
            \includegraphics[width=1\textwidth]{graphics/trangle_scan_genkt}
        \end{minipage}\hfill
        \begin{minipage}[c]{0.35\textwidth}
            \caption{The \genkt{} algorithm has 2 parameters that can be varied.
                The stopping condition, \ktstoppingdeltar{}, and a multiple for the exponent of the \(p_T\) factor.
                When the exponent of the \(p_T\) factor is \(-1\) the algorithm becomes the \antikt{} algorithm.
                Here, the ``Loss'', as described in Eq.~(\ref{eqn:loss}), is shown as a colour gauge for a number of parameter combinations.
             }\label{fig:scan_genkt}
        \end{minipage}
    \end{figure}    

There are two ways a jet finding algorithm can make mistakes in this task:
the first is to omit some of the descendants of the objects being reconstructed, causing the jet to have less mass than it should;
the second is to include particles that are not in the descendants of the objects being reconstructed, such as initial state radiation or particles from other objects,
causing the jet to have more mass than it should.
The effects of these mistakes will cancel in the jet mass,
but they are both still individually undesirable,
so separate metrics are made for each of them.
The first is ``signal mass lost", the difference between the mass of the jets and the mass they would have had if all they contained all descendants of the object being reconstructed.
The second is ``background contamination", the difference between the mass of jets and the mass they would have if they did not contain anything but descendants of the objects being reconstructed.
A loss function is constructed as a weighted euclidean combination of these two;
\begin{equation}\label{eqn:loss}
\text{Loss} = \sqrt{w\,(\text{Background contamination})^2 + (\text{Signal mass lost})^2}.
\end{equation}
Where \(w\) is a weighting used to alter the preference for suppressing signal mass lost verses reducing background contamination.
When applying an \antikt{} algorithm, increasing \ktstoppingdeltar{} will result in lower signal mass loss, in exchange for a higher background contamination.
The standard choice for this process is \(\ktstoppingdeltar{}=0.8\).
This value of \ktstoppingdeltar{} slightly prefers suppressing signal mass lost over background contamination,
to create the clearest mass peaks.
To make the loss reflect this we chose \(w = 0.73\).

An example of this scan for the \genkt{} algorithm is given in Fig.~\ref{fig:scan_genkt}. 
It can be seen that, while good results are possible with many values of the \(p_T\) exponent,
                \ktstoppingdeltar{} must fall in a narrow range. We thus deem this choice of  stopping condition, \(\stoppingdeltar{}_{k_T}=0.8\), to be rather fine-tuned.

For \spectral{} clustering there are more than 2 variables to deal with, 
so a set of two dimensional slices are extracted. 
These slices have been chosen to include the best performing combination.
%
    As can be seen in Fig.~\ref{fig:scan_spectral}, the parameters choices  are not fine-tuned. That is, 
    unlike the \antikt{} algorithm, there is flexibility in all parameter choices. For example, it can be seen that some parameters, such as \(\alpha\), \(k_\text{NN}\), \(\beta\)
    and \(\lambda_\text{limit}\) are relatively unconstrained,
    yielding  good results for a wide range of values.
    Even when \stoppingdeltar{} and, especially, \(\sigma_v\) yield some large signal losses,
    say, for \(R=1.22\) or \(1.3\) and \(\sigma_v=0.05\), this happens is very narrow ranges. 
    For definiteness, the
    parameters used in the remainder of this work are \(\alpha=2.\), \(k_\text{NN}=5\), \(\stoppingdeltar{} = 1.26\), \(\beta = 1.4\), \(\sigma_v = 0.15\) and \(\lambda_\text{limit} = 0.4\).
\clearpage
    \begin{figure}[!t]
            \includegraphics[width=1\textwidth]{graphics/trangle_scan_complete}
            \caption{There are 6 free parameters in \spectral{} clustering.
                Here, the ``Loss'', as described in Eq.~(\ref{eqn:loss}), is shown for reasonable  parameter ranges chosen
                either by convention (e.g., \(\alpha\) is typically \(1\) or \(2\))
                or according to physical scales (e.g., $\sigma_v$ is of order \(0.1\)).
             }\label{fig:scan_spectral}
    \end{figure}    


    \subsection{Particle data}\label{sec:particle_data}

To evaluate the behaviour of the spectral clustering method four datasets are used\footnote{The first two uses a 2-Higgs Doublet Model (2HDM) setup as described in Ref.~\cite{Chakraborty:2020vwj} while the last two are purely Standard Model (SM) processes. Notice that all unstable objects are rather narrow, including the Beyond the SM (BSM) Higgs states \cite{Moretti:1994ds,Djouadi:1995gv}, so that we have neglected interference effects with their irreducible backgrounds.}, all produced for the Large Hadron Collider (LHC).

    \begin{enumerate}
        \item 
    \underbar{Light Higgs} A SM-like Higgs boson with a mass \(125\) GeV decays into two light Higgs states with mass  \(40\) GeV,
    which in turn decay to \beau{}\bbar{} quark pairs.
    That is, the process is \(p p \rightarrow H_{125\,\text{GeV}} \rightarrow h_{40\,\text{GeV}} h_{40\,\text{GeV}} \rightarrow \beau \bbar \beau \bbar\), simulated at Leading Order (LO).

\item \underbar{Heavy Higgs}  A heavy Higgs boson with a mass \(500\) GeV decays into two SM-like Higgs states with mass  \(125\) GeV,
    which in turn decay to \beau{}\bbar{} quark pairs.
    That is, the process is \(p p \rightarrow H_{500\,\text{GeV}} \rightarrow h_{125\,\text{GeV}} h_{125\,\text{GeV}} \rightarrow \beau \bbar \beau \bbar\), simulated at LO.

\item \underbar{Top}  A $t\bar t$ pair decays semileptonically, i.e., where one \(W^\pm\) decays to a pair of quark jets $jj$ and the other into a lepton-neutrino pair $\ell\nu_\ell$ ($\ell=e,\mu$).
        That is, the process is \( p p \rightarrow t \bar{t} \rightarrow b\bar b W^+  W^-\to b\bar b jj \ell\nu_\ell\), simulated at LO. (Note that, here, $m_t=172.6$ GeV and $m_{W}=80.4$ GeV.)
        
    \item \underbar{3-jets}  For the purpose of checking {IR safety}, we have used three-jet events,         this being a rather simple configuration where IR singularities could be observed. 
        That is, the process is $pp\to jjj$, simulated at both LO and Next-to-LO (NLO). 



    \end{enumerate}

    Using MadGraph~\cite{Alwall:2011uj} to generate the partonic process and Pythia~\cite{Sjostrand:2014zea} to shower, ${\cal O}(10^5)$ of each of these processes are generated.
    A full detector simulation is not used; instead, cuts on the particles are imposed to approximate detector resolution, as detailed below. 
    
    The Center-of-Mass (CM) energy used is \(\sqrt{s}=13 \) TeV.

    Each event also contains (hard) Initial State Radiation (ISR) and soft QCD dynamics from beam remnants, i.e., the Soft Underlying Event (SUE).
    There is no pileup or multiparton interactions in the datasets.

    Each of these datasets requires different cuts, both at the particle level, to simulate detector coverage, and at the jet level, to select the best reconstructed events.
    The cuts on each dataset are as follows.
    \begin{enumerate}
        \item The reconstructed particles are required to have
            rapidity \(|\eta|< 2.5\) and (transverse momentum \(p_T > 0.5\) GeV.
            These cuts are likely to remove the majority of the radiation from beam remnants
            and reduce the radiation from ISR.
            The $b$-jets are required to have \(p_T > 15\) GeV, which is possibly lower than is realistic \cite{Chakraborty:2020vwj},
            but it leaves a larger number of events to compare the behaviour of jet clustering algorithms.

        \item  The reconstructed particles are required to have
             \(|\eta|< 2.5\) and \(p_T > 0.5\) GeV.
            The $b$-jets are required to have \(p_T > 30\) GeV, which is realistic for efficient $b$-tagging performance and further reduces ISR and the SUE.
            
        \item The reconstructed particles are required to have
             \(|\eta|< 2.5\) and \(p_T > 0.5\) GeV.
            The event is required to have  \(p_{T}^{\text{miss}} > 50\) GeV,
            where \(p_{T, \text{miss}}\) is the missing transverse momentum due to 
            the neutrino.
            The lepton in the event must have  \(|\eta|< 2.4\).
            If the lepton  is a muon then its \(p_T\) must be \(>  55\) GeV.
            If the lepton  is an electron and it is isolated (as defined in~\cite{Sirunyan:2018fpa}) then its \(p_T\) must be \(> 55\) GeV, if it is not isolated then \(p_T > 120\) GeV.
            The reconstructed jets must have \(p_T > 30\) and \(|\eta|< 2.4\).
            Finally, the lepton must be separated from the closest jet by at least
            \(\sqrt{\Delta\eta^2 + \Delta \phi^2} > 0.4\) or
            \(p_{T}^{\text{relative}} > 40\) GeV.
            These cuts are copied from~\cite{Sirunyan:2019rfa}.
        \item The only restriction on the particles is that the rapidity must be \(< 2.5\).
            There are no cuts on the jets. While unrealistic, since 
            issues of IR safety are emphasised at low \(p_T\),  to highlight this we abandon all \(p_T\) cuts.

    \end{enumerate}


    The Higgs boson cascade datasets have the desirable property of creating \bthing{jets} with different kinematics: while in case 1 we may expect some slim  jets (as on average they are rather stationary, because of the small mass difference between $H_{125\,{\rm GeV}}$ and $H_{40\,{\rm GeV}}$)
in case 2 we may see mainly fat jets (owing to the boost provided by the large mass difference between $H_{500\,{\rm GeV}}$ and $H_{125\,{\rm GeV}}$).
Mass reconstruction requirements for the \underbar{Light Higgs} and \underbar{Heavy Higgs} follow the same logic.
In order to reconstruct a Higgs decaying directly to a pair of \bthing{quarks}, we require a separate jet tagged by each \bthing{quark}, that is, two jets are required, each tagged by a \bthing{quark} from that Higgs.
To reconstruct a Higgs that decays to a pair of child Higgs particles,
we require both child Higgs bosons have been reconstructed,
that is, all four \bthing{jets} are found.
In the case of the \underbar{Top} events
three masses can be reconstructed from jets, the hadronic \(W\),
the hadronic top and the leptonic top.
The hadronic \(W\) is reconstructed if both of the quarks it decayed to have tagged jets; they are permitted to tag the same jet, so the hadronic \(W\) can be reconstructed from one or two jets.
The hadronic top is reconstructed if the hadronic top is reconstructed and the \bthing{quark} from the hadronic top has tagged a jet, so the correct \bthing{jet} is required in addition to the requirements on the \(W\).
The leptonic top is reconstructed if the \bthing{quark} from the top decay tags a jet, and the missing momentum calculation which reconstructs the leptonic \(W\) yields a real mass.
If the missing mass calculation for the mass of the leptonic \(W\) yields two real masses, the one closes to the \(W\) mass is selected.

We now proceed to compare spectral to anti-$k_T$ clustering and we start from testing IR safety of the former, while this is a well-known feature of the latter. We will then move on to study Higgs boson and top quark events.


\subsection{Determining IR safety}\label{sec:IRmethod}
    It would be possible to demonstrate IR safety analytically, however,
    as the environment required for clustering on MC data is already set up,
    it is more efficient for this study to prove IR safety with such data.
    This can be done by showing that an IR sensitive variable, for example, the jet mass spectrum,
    is stable between a LO dataset with no IR singularities and a NLO
    one which will instead contain IR singularities.

    Showing the jet mass spectrum at LO and NLO for a particular configuration,
    that is, a particular selection of clustering parameters,
    would allow a comparison that would highlight any differences caused by IR sensitivity.
    This will be done for illustrative purposes,
    however, even an IR unsafe algorithm, such as the iterative cone one~\cite{Cacciari:2008gp},
     has some configurations for which these singularities are avoided.

    To provide a more global view, a scan of parameter configurations must be compared.
    Thus, for an unsafe algorithm (such as the iterative cone) the unsafe configuration
    will be found.
    It would be cumbersome to compare all these jet mass spectrum by eye, however.
    Instead, we introduce a summary statistic representing the divergence between two distributions,
    the Jensen-Shannon score~\cite{Lin:1991zzm}.

    The Jensen-Shannon score is a value computed between two distributions that increases in magnitude the more these distributions differ.
    It is a symmetrised variant of the Kullback-Leibler divergence~\cite{Lin:1991zzm}.
    The Kullback-Leibler divergence between probability densities \(p\) and \(q\) can be written as
    \begin{equation}
    D_\text{KL} (p | q) = \int^{\infty}_{-\infty} p(x) \log\left(\frac{p(x)}{q(x)}\right) dx,
\end{equation}
    from which the Jensen-Shannon divergence can be written as
    \begin{equation}
    D_\text{JS}(p, q) = \frac{1}{2}D\left(p | \frac{1}{2}(p + q)\right) + \frac{1}{2}D\left(q | \frac{1}{2}(p + q)\right).
\end{equation}
    Here, \(D_\text{JS}\) treats \(p\) and \(q\) symmetrically and will grow as they become more different.
    The spectrum of Jensen-Shannon scores will be plotted for a known IR safe clustering algorithm, \antikt{},
    a known unsafe clustering algorithm, iterative cone, and the \spectral{} algorithm.
    If the Jensen-Shannon scores for \spectral{} are consistently small,
    then it is IR safe.


    \FloatBarrier
    \subsection{Results}
This section is still subject to change as findings reveal advantages and drawbacks
of parameter choices.
Three choices of clustering algorithm are considered here;
\begin{itemize}
    \item A standard anti-KT algorithm with \(\stoppingdeltar{}=1.36\)
    \item A \spectralfulljet{} algorithm with \(2\) eigenvectors, \(\stoppingdeltar{}=0.63\), and \(q=-0.035\)
    \item A \spectralmeanjet{} algorithm with \(4\) eigenvectors, \(\stoppingdeltar{}=0.70\), and \(q=0\)
\end{itemize}

As the data is simulated it is possible to compare the performance of clustering algorithms to Monte Carlo truth.
Each event contains 4 \bthing{quarks} and for each of them it is possible to identify the particle into which they decayed, as a subset of the particles in the final state.
Henceforth the detectable decay products of the \bthing{quark} will be called the descendants of the \bthing{quark}.
For two reasons it is not possible for this clustering algorithm to gather all the descendants
of each \bthing{quark} into one jet:
firstly not all the descendants make the \(p_T\) and \(\eta\) cuts, so some are discarded  before clustering;
secondly the descendants of the \bthing{quarks} in an event are not mutually exclusive, due to interactions during hadronisation the quarks share descendants, and our clustering algorithm does produce exclusive clusters.

Knowing the parts of the final state that are descended from each \bthing{quark} creates a clear
allocation of jets to quarks.
For each quark, the jet that contains the greatest mass in descendent particles is tagged to represent that quark.

Mass peaks can the be constructed from the tagged jets, using all the particles in the jet,
both descendant of the quarks and background.
In figure~\ref{fig:best_all} the masses of all jets that have been \bthing{tagged} are plotted.
In figure~\ref{fig:best_correct_h_allocation} three selections are plotted; firstly only events where some trace of all 4 \bthing{quarks} is found
are plotted with the mass of the descendants of the heavy Higgs in the background.
Then the two light Higgs in each event are sorted by the mass of their descendants,
in effect they are ranked by how well they were picked up by the detector.
The mass of the light descendants is plotting in the background and over
that the mass of the associated jets in each event is shown.


\begin{figure}[htp]
    \begin{minipage}[c]{0.5\textwidth}
        \includegraphics[width=1\textwidth]{graphics/best_all.png}
    \end{minipage}\hfill
    \begin{minipage}[c]{0.45\textwidth}
        \caption{All tagged jets in each event are plotted.
            A peak just shy of \(125\)GeV is desired as this is the mass of the decaying
            heavy Higgs.
            Another peak at \(40\)GeV could be observed is only one light Higgs
            was found in the detector volume.
            \spectralfulljet{} gets closest, peaking close to \(115\) GeV and
            showing some contamination from background with the bins \(>125\) GeV.
            Anti-KT and \spectralmeanjet{} both give similar behaviour, falling somewhat short.
            The Anti-kt jet used a \stoppingdeltar{} of \(0.63\).
        }\label{fig:best_all}
    \end{minipage}
\end{figure}    


\begin{figure}[htp]
    \includegraphics[width=1.\textwidth]{graphics/best_correct_h_allocation.png}
    \caption{Starting with the plot on the far left, the jet mass of events where
        some descendants from all 4 \bthing{quarks} were found is plotted over
        the total mass of all descendants.
        The plot in the centre takes the light Higgs whose descendants have the greatest mass in each event.
        The mass of these descendants is plotted in grey, and over this
        the masses of the jets in each event that correspond to this better observed Higgs
        are shown.
        That is, the tags in each event that have been produced by the better observed light Higgs
        are allocated to jets, and only the mass of these jets is tabulated - thus
        a good cluster will have obtained the mass of the light Higgs descendants.
        Finally on the left the light Higgs who's descendants have less mass is shown along with
        the corresponding jets.
        These three plots make it clear that only \spectralfulljet{} is obtaining the majority of the
        descendants, but it is combining them with some background as the tail of the
       \spectralfulljet{} distribution continues past the descendants distribution
       out of the frame.
       The behaviour of \spectralmeanjet{} does not significantly differ from Anti-KT
    }\label{fig:best_correct_h_allocation}
\end{figure}    

    \FloatBarrier
    \subsubsection{Going forward}
From here, there a two possible ways forward;
\begin{enumerate}
    \item Introduce some \(p_T\) dependence.
        Due to the high parameter space this will require writing in
        an optimiser.
        Not conceptually difficult, but time consuming.
    \item Investigate Diffusion mapping.
        This has some resemblance to \ref{compressionstep} in the current method.
        It might also provide a faster algorithm.
\end{enumerate}

I am inclined to try 1 first then 2
because I am confident 1 will yield results.

    \FloatBarrier
    %\section{Acknowledgements}
    %I would like to thank Professor Stefano Moretti, Professor Claire Shepherd-Themistocleous, Professor Srinandan Dasmahapatra, Dr Emmanuel Olaiya, Prof. Rachid Benbrik and Dr Amit Chakraborty for their excellent supervision and advice.
    %My colleagues, Souad Semlali, Billy Ford and Shubhani Jain have been very generous in their assistance and collaboration.
    %I would also like to thank Dr Mauro Verzetti for assistance with the data and Dr Jim Pivarski for creating the excellent packages \lstinline{uproot} and \lstinline{awkward-array}.
    %The IRIDIS High Performance Computing Facility was used for this work,
    %and I am grateful for the assistance of associated support services at
    %the University of Southampton.
    \printbibliography	
\end{document}
