\documentclass{article}
\usepackage{geometry}
\newgeometry{vmargin={18mm}, hmargin={20mm,20mm}}
\usepackage[export]{adjustbox}[2011/08/13]
\usepackage{subcaption}
\usepackage{placeins}
\usepackage[backend=biber]{biblatex}
\usepackage{graphicx}
\usepackage{xcolor}
\usepackage{listings}
\lstset{basicstyle=\ttfamily}
\usepackage{caption}
\usepackage{amsmath}
\usepackage{siunitx} %This is currently a prealpha version that I got by accident
\let\vec\mathbf
\newcommand{\homejet}{\lstinline{Traditional}}
\newcommand{\fastjet}{\lstinline{FastJet}}
\newcommand{\spectralmeanjet}{\lstinline{SpectralMeanJet}}
\newcommand{\spectralfulljet}{\lstinline{SpectralFullJet}}
\newcommand{\splittingjet}{\lstinline{SplittingJet}}
\newcommand{\indicatorjet}{\lstinline{IndicatorJet}}
\newcommand{\stoppingdeltar}{\ensuremath{\Delta R}}
\newcommand{\distancedeltar}{\ensuremath{\Delta R'}}
\newcommand{\beau}{\ensuremath{b}}
\newcommand{\bbar}{\ensuremath{\bar{b}}}
\newcommand{\bthing}[1]{\ensuremath{b\text{-#1}}}
\bibliography{writeup}
\graphicspath{{./graphics/}}
\begin{document}
	\title{Clustering jets}
	\author{Henry Day-Hall}
	
	\maketitle
	
    %\begin{abstract}
    %    Stuff
	%\end{abstract}
    
	\tableofcontents
    \FloatBarrier
    \textcolor{red}{
        \section{To do}
        \begin{itemize}
            \item Try Spectral Mean with a conductance condition for cut off. Written, need to run and test.
            \item Try splitting with multiple eigenvectors.  Not sure how?
            \item Look at scores just on events that have 3 tags in range.  % after rescoring
            \item Check assertions about spectral clustering behaviour.   % after rescoring
            \item Try to get some theory behind what should work. See sec \ref{sec:ideas}.  % after rescoring
                \begin{itemize}
                    \item Looked at distances in eigenspace - results indicate should be normalizing.
                    \item Looked at distance in physical space - Results indicate Luclus is a good idea
                \end{itemize}
            \item Calculate the Rand Index to compare spectral and traditional methods.
            \textcolor{black}{\item Do b quarks ever interact with b quarks from the other higgs? No, they always share with the same parent.}
            \textcolor{black}{\item Change tagging procedure; the centre of mass energy will tend
            to favour particles that have large momentum. Done, need to rescore. Done}
        \end{itemize}
    }
    \FloatBarrier
	\input{hepPhenoOverview}
    \FloatBarrier
    \input{dataset}
    \FloatBarrier
	\input{method}
    \FloatBarrier
    \input{mc_plots}
    \FloatBarrier
    \input{shape_plots}
    %\section{Acknowledgements}
    %I would like to thank Professor Stefano Moretti, Professor Claire Shepherd-Themistocleous, Professor Srinandan Dasmahapatra and Dr Emmanuel Olaiya for their excellent supervision and advice.
    %I would also like to thank Dr Mauro Verzetti for assistance with the data. 
    %The IRIDIS High Performance Computing Facility was used for this work,
    %and I am grateful for the assistance of associated support services at
    %the University of Southampton.
    \printbibliography	
\end{document}
